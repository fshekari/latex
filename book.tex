\documentclass{book}
\usepackage{multicol}
\usepackage{graphicx}
\usepackage{multicol}
\usepackage{geometry}
\usepackage{array}
\usepackage{float}
\begin{document}
	
	\begin{table}
		\begin{center}
			
		\caption{\textbf{Table 5.3 Software Planning Tasks}}
		\label{Table 5.3}
		\centering
		\setlength{\leftmargini}{0.5cm}
		\begin{tabular}{| m{8cm} | m{8cm} |}
			
			\hline Software Planning Tasks & Locations\\
			\hline Methods for developing and maintaining the SDP & 1.5.1, 5.1.1, 5.3.2, 6.6.3, 10.1, 13.3\\
			\hline Software Data Management (DM) & 4.4, 5.1.10, 10.2, 9.5, 12.5, 13.4\\
			\hline Software size and resource estimation & 3.2.2, 3.4, 3.7, 5.1.3, 9.5.1, 13.4, Chapter 14\\
			\hline Software build planning & 3.4, 5.1.4, 8.2, 8.7\\
			\hline Software integration and test (I\&T) planning & 2.3.9, 3.6, 3.7, 10.3, 8.5, 8.7\\
			\hline Software Development Environment and support tools & 6.3, 12.1, 12.3, 12.11\\
			\hline Software acceptance, delivery, installation, transition, operations,\\
			sustainment and retirement planning  & 5.1.9, 8.8, 8.9, 9.8.2, 16.2, 16.3\\
			\hline Software Configuration Management & 2.4.3, 5.1, 6.3, 6.4, 8.6, 12.3, 16.1\\
			\hline Software evaluations with formal and informal reviews & 2.4.5, 6.1.2, 6.2, 6.5, 13.4\\
			\hline Software Quality Assurance (SQA) & 2.3.5, 2.4.2, 5.1, 6.2, 10.8.2, 13.4\\
			\hline Problem resolution methods and Preventive Action & 5.1, 5.2.3, 6.2, 6.4, 10.3.4, 10.4.5, 8.5.8, 9.10.5, 12.4, 15.2\\
			\hline Software risk management & 1.6.3, 2.3.2, 5.2, 6.4.3, 6.6.3, 9.5.2, 9.6.3, 13.4, 16.5\\
			\hline Software metrics covering products and processes & 1.5.1, 2.3.3, 3.10.8, 5.1.4, 6.1.2, 13.4, Chapter 15\\
			\hline Security and Privacy issues & 3.10.2, 5.2.5, 5.4.1, 10.4.3, 9.5, 9.8.2\\
			\hline Oversight of software subcontracts & 5.1.2, 5.5, 11.2, 11.3.2\\
			\hline Software schedules with critical interdependencies & 2.3.4, 2.6, 4.4, 5.1, 8.6, 8.8, 9.8.2, 11.4, Appendix D\\
			\hline Software organization, roles and responsibilities & 1.5.1, 1.6, 8.1.4, 8.2.3, 8.4.3, 8.5.3, 8.6.3, 10.1, Appendix C\\
			\hline Required resources, skills and staffing plan & 10.1, 10.2, 10.3, 11.3\\
			\hline Training plans and training requirements & 2.3.7, 4.3.9, 6.1.3, 8.9.9, 10.5, 10.8.4, 10.9, 11.3.5\\
			\hline Software Operations and Maintenance & Chapter 16\\
			\hline
		\end{tabular}
		\end{center}
	\end{table}
	\begin{multicols}{2}
		\textbf{Corrective Action Process (CAP)}
		
		Unplanned updates to the SDP must be handled through the CAP as described
		in the SDP or the Corrective Action Plan if you have one. All changes to the SDP should require approval by the CSwE
		and the program SEPG. This Guidebook assumes the project has a CSwE, or an equivalent position, as described in
		Subsection 1.6.4. Changes to the SDP also should require CCB approval as discussed in Subsection 6.4.3. The SEPG is
		described in Subsection 6.1.1.
		
		\textbf{Software Entity Database (SWED)}
		
		A database, that can be called the Software Entity Database, should be produced and periodically updated to provide a mechanism for identifying, profiling and tracking all Software Items (SIs) on
		the project. Each subsystem should be responsible for their data input to the SWED, but the CSwE should be responsible for compiling this information into a single centralized and controlled database for the project or the program. This database may include for each SI in the project: a functional description of the SI, class, category, size, the percent of new
		versus reused code, responsible Developer(s) and their contact information, and language(s) used.
		
		\textbf{5.1.3 Software Resource Estimating} 
		
		Software resources, including physical, personnel, cost and computer resources, should be estimated before software
		development begins. These estimates are used to establish software development schedules, risk mitigation plans, and
		commitments and should be documented in a Software Resource Estimation Plan. Details of managing the software
		resources are provided in Part 3 of the Guidebook.Software personnel should participate with other affected
		groups (Systems Engineering, SQA, SCM, test, etc.) in the overall program planning throughout the program as  %page 2

		members of Software Integrated Product Teams (SwIPTs).
		The SwIPTs are discussed in Subsection 1.6.5. Commitments,
		or changes to commitments, made to individuals and external groups must be reviewed with management regularly.
		\textbf{Staffing Estimation.} In order to determine the level
		of staffing required, the planning function should consider
		program constraints including milestones, reviews, document and product deliveries, internal milestones, incremental builds, technical constraints, and any changes in
		scope. Estimates of Source Lines Of Code and software
		development productivity play an important role in staffing
		estimates.
		\textbf{Re-planning}. The software groups should participate,
		when required, in re-planning activities to address contract
		changes, process improvements, or when measured performance varies from planned performance. The related data
		that is generated must be maintained and placed in the
		applicable Software Development Files (SDF) or Software
		Engineering Notebooks. Software personnel also should participate in contract/subcontract modification activities (such
		as engineering change proposals). Chapter 14 covers software
		estimating fundamentals, methods, size growth and estimation tools.
		\textbf{\textit{5.1.4 Software Build Planning}}
		A software build is a portion of a system that satisfies, in part
		or completely, an identifiable subset of the total end-item or
		system requirements and functions. There are often multiple
		internal builds leading to a deliverable build for an increment
		in the life cycle. Requirements met in one incremental build
		are also met in all successive increments. The final build is the
		complete software system. A release is a build version that
		is delivered for acceptance testing and subsequently may be
		released or delivered for operational use. Incremental builds
		can be planned for each SI, or group of SIs.
		\textbf{Software Build Plan.} A table, similar to the example in
		Table 5.4, must be included in the SDP, or a separate document referenced by the SDP, to show the intended software
		delivery plan. The table must include a unique number, often
		called the Program Unique Identifier (PUI), for each Software
		Item and its name, the responsible developing organization,
		and Equivalent Source Lines of Code (ESLOC) planned for
		each build. Subsection 14.1.3 provides an explanation of
		how ESLOC is derived. As shown in Table 5.4, the version
		(preliminary, initial, update, fixes) can be identified for each
		delivery.
		\textbf{Software Master Build Plan (SMBP)}. A comprehensive
		SMBP must be provided to map the incremental functionality, capabilities and requirements allocated to each build.
		The CSwE, or Build Manager, usually maintains the SMBP
		with the approval of the Software CCB. Once approved, the
		SMBP should be controlled by SCM. In a large program,
		this is not a trivial issue.
		As the Project Manager, you, the CSwE, or the Software
		Build Manager, should routinely report the status and 

		
	\end{multicols}
 
	\begin{table}
		\begin{center}
			
		\caption{\textbf{Table 5.4 SI Build Delivery Plan—Example}}
		\label{Table 5.4}
		\begin{tabular}{|l |l |l |l |l |l |l|}
			\hline PUI & Software Item Name & Developer & Build-1 & Build-2 & Build-3 & Total\\
			\hline    & Total for 18 SIs: &  & 102,000 & 65,000 & 130,000 & 297,000\\
			\hline 1.0 & Decision Support & & 45,000 & 28,000 & 48,000 & 121,000\\
			\hline 1.1 & Decision Analysis & Able Corp & I & UWDR & U & \\
			\hline 1.2 & Analytical Algorithms & Able Corp &  & I & U & \\
			\hline 1.3 & Scenario Analysis & Able Corp &  & I & U & \\
			\hline 1.4 & Test Bed Controls & Baker Co. & P & I & U & \\
			\hline 1.5 & Traffic Control & Baker Co. & I & UWDR & U & \\
			\hline 1.6 & Simulation Analysis & Baker Co. & P & I & U & \\
			\hline 2.0 & Services Support & & 30,000 & 18,000 & 12,000 & 60,000\\
			\hline 2.1 & Routing Analysis & Charlie Co. & P & I & U & \\
			\hline 2.2 & User Support & Charlie Co. &  & I & UWDR & \\
			\hline 2.3 & XYZ Services & Charlie Co. & I  & UWDR & UWDR &\\
			\hline  & etc. & etc.& & & & \\
			\hline
		\end{tabular}
		\end{center}
	\end{table}
	
	\textbf{P} = Preliminary Version; \textbf{I} = Initial Delivery; \textbf{U} = Updated Delivery; \textbf{} = No Delivery; \textbf{UWDR}= Updates for Work-Offs of
	Discrepancy Reports;\textbf{ “00,000”} = Number of ESLOC per Build; \textbf{PUI} = Program Unique Identifier.%page 3
	\begin{figure}[h]
		\centering
		\includegraphics[scale=0.6]{pic1.jpg}
		\caption{Software management from a measurement perspective}
		\label{pic1}
	\end{figure}
	\begin{multicols}{2}
		changes to the involved upper-level Program Manager(s). The
		SMBP may also be called the “Master Software Integration
		and Verification Plan” or may be referred to as a “Build
		Functionality Matrix.” The SMBP is a software document,
		and it must be compliant with the Master Test Plan (MTP)
		which is a system-level document covering hardware and
		software.
		\textbf{Build Planning Updates}. Software build planning
		should occur for each program increment and each deliverable build and be updated continuously throughout the program. Build plans are typically updated only when the plan
		contents change significantly as determined by the SwIPT
		Lead.
		When planning the builds, the project schedule, ESLOC,
		and functional content estimates must be taken into consideration. As the program matures, additional design, requirements, technical content, and testing approaches should be
		added. The build activities should be documented in detailed
		schedules and then incorporated into the IMS along with
		staffing and budget-plan information.
		\textbf{5.1.5 Software Development
			Tracking and Oversight}
		The software tracking and oversight effort begins once software planning is complete. Subsystem SwIPTs, Software
		Leads, the Chief Software Engineer, and Software Quality
		Assurance, can monitor software development status by:
		\begin{itemize}
		\item Collecting and evaluating software measurement data
			(see Chapter 15)
		\item Performing product quality and process audits (see
			6.1.2 and 6.2)
		\item Supporting software reviews (see 5.3)
		\item Performing risk management activities (see 2.3.2, 3.9.4
			and 5.2) 
		\end{itemize}
		
		\textbf{Software Measurement Oversight}. Throughout the
		development process, software measurement data must
		be used to compare actual software size, cost, schedule
		and progress against the established plan so that you
		can take timely Corrective Actions (see Section 6.4). If
		those metrics indicate out-of-tolerance conditions, you or
		Subsystem SwIPT members must perform an analysis to
		identify the problem. At that point, you must take the
		appropriate Corrective Action and identify the potential
		risks including cost and schedule impacts. The Software
		Measurement (or Metrics) Plan is an important addendum
		to the SDP. Chapter 15 is entirely devoted to software
		measurements.
		The status of software should be reviewed (usually
		weekly) at subsystem-level meetings and at monthly program
		status meetings. In addition, software status should be provided to the customer monthly and also at quarterly reviews.
		Software management and control must be integrated into
		your overall program management scheme. Figure 5.2 is a
		depiction of software management from a measurement
		perspective.
		As shown in Figure{\ref{pic1}}, software measurement oversight
		involves:
		\textbf{Software-specific measurements:}
		\begin{itemize}
		\item Schedule and Progress Measurements
		\item Resources and Cost Measurements
	    \item Product Quality Measurements
		\item Performance Measurements
		\item Growth and Stability Measurements
		\end{itemize}
		
		\textbf{Non-software-specific measurements:}
			\begin{itemize}
		\item Risk Management Measurements
    	\item Financial Management Measurements
		\item Programmatic Measurements %page4
			\end{itemize}
		\textbf{Cost Account Oversight.} Software work packages are
		typically cost accounts within the Earned Value Management
		System (EVMS) used by many contractors (EVMS is
		described in Section 8.5). The cost account must be at a level
		of detail sufficient to maintain control of the associated software development activities. Ongoing metrics collected on
		the cost accounts and work packages must be reported to
		program management and available to the customer.
		\textbf{Schedule Oversight.} Schedule review meetings are normally conducted weekly or more often if the schedule is consistently changing. Schedule metrics (using weekly milestone
		accomplishments, including subcontractor data) should be
		reported along with the status of Corrective Action/recovery
		plans. IMS and detailed schedules should also be reviewed at
		lower levels within the SwIPTs.
		\textbf{Headcount Oversight}. SwIPTs should monitor headcount on a weekly basis and strive to identify potential problems early. Updates of accomplishments, actual budgeted
		and forecasted headcounts should be conducted monthly.
		Forecasts should be updated and reported in internal cost
		performance reports that include cost/schedule variances
		and changes in the latest revised estimate. Costs and schedules should be controlled by monitoring headcount, expenditures, and assessing progress.
		\textbf{Product and Process Oversight}. Product evaluations,
		software reviews, process audits and assessments are used by
		Subsystem SwIPTs, CSwE, and SQA as a means to determine
		compliance with the standards established by the SDP. Noncompliance of baselined products is handled via the Corrective
		Action Process (see Section 6.4). Process audits must be performed by SQA, with support from the CSwE, to determine
		compliance with the processes specified in the SDP. SQA
		must be responsible for documenting and verifying closure
		of a non-compliance issue. Subsection 6.5.6 describes the
		software product evaluation process. You or your senior management must implement and maintain the mechanisms for
		interfacing to and communicating with the customer.
	\end{multicols}

	\begin{figure}[h]
	\centering
	\includegraphics[scale=0.7]{pic2.jpg}
	\caption{Relative cost of gold plating.}
	\label{pic2}
\end{figure}
\begin{multicols}{2}
\textbf{5.1.6 Avoid Gold Plating}
A passion for excellence and efficiency is much less
expensive to attain than a passion for perfection.
There is a tendency for all Engineers to overachieve. That
is often called “gold plating.” Tasks performed that are not
called out in the contract as a system requirement can be
considered gold plating unless the task is a derived requirement essential to system functionality. Rework to make it
“better” and then more rework to make it “even better,” and
continuing this “polishing the apple” approach always has a
severe impact on project cost and schedule. It is understood
that Engineers typically want to continue working on their
task to make it as good as possible. As the Software Project Manager, you must be careful to avoid this perfectionist pitfall.
For example, Figure(\ref{pic2})is a notional curve showing the relative cost of perfection versus system functionality required.
The cost and time to achieve perfection is very high.
Where human lives and safety are involved, perfection
can be justified, and the customer must pay for it; otherwise, perfection is too expensive. When you reach 100% of the
functionality required by the contract, you have fulfilled the
contract requirements. If the time and cost at that point is
\$1X, as shown in Figure(\ref{pic2}), perfection can be at least three
times more expensive and time-consuming.
\textbf{Lessons Learned}. The last 5\% is costly. It may be
human nature to delay submitting your product
until it is perfect, but it is a Software Manager’s
responsibility to see to it that gold plating does
not happen. I have observed that, if human lives
and safety or not involved, 95\% of the planned
functionality often satisfies 95\% of the users;
that last 5\% is a very steep part of the curve so, if
you can avoid it, you could realize a big cost savings. This is a tricky game to play. Most “shrink
wrapped” software products probably do not
come close to 95\% of full functionality when the
product is initially delivered to the marketplace.
\textbf{5.1.7 Software Item Test Planning}
The testing of Software Items being developed by the subsystems should be performed by the respective subsystem Software Test Engineers. They are responsible for documenting their Software Test Plan (STP), Software Test
Description (STD), and Software Test Report (STR) to verify
that the SIs meet their allocated subsystem requirements.
A preliminary version of the STP is usually produced
during Software Design activities, but the full STP is the
result of the SI test planning activity. Production of the STD
and STR is performed during Software Item Qualification
Testing (SIQT) and is discussed in Section 11.7. Test activities for the critical (SCS) and support (SS) software classes
must also be documented in the Software Development Files
as described in Section 9.4.
\textbf{Software Test Plan.} The STP describes plans for qualification testing of SIs and is an important software document.
It describes the test environment to be used, identifies the
tests to be performed, provides schedules for the testing tasks,
defines the resources needed, and addresses the planning
tasks required to conduct the SIQT. Table 5.5 is an example
summary of the readiness criteria in terms of entry, exit and
verification criteria to ensure completeness of the STP.
\textbf{5.1.8 System Test Planning}
The SDP should contain the approach for providing support
to System Test Planning. If there is a System Verification and
Test Plan (it may also be referred to as an Integration, Test and
Evaluation Plan), it should be prepared by the SEIT. It is the key
planning document for system testing. System Integration and
System Test activities are described in sections 10.3 and 10.4.
The System Test Team should be responsible for performing the actual system testing. Software Developers and/or
Software Test Engineers have a support role in System Test
Planning that may include reviewing test preparation materials and providing software test support items, such as reusable software test documentation, simulators, drivers and
analysis tools. Software Engineers also support anomaly
analysis to determine if the problem is due to software only,
hardware only, or a combination. If Regression Testing on
the software builds is needed, SCM must provide the software builds against which the tests are conducted. Regression
Testing is discussed in Subsection 3.9.3.
\textbf{5.1.9 Planning for Software Transition}
to Operations and Maintenance
Planning for Software Transition to Operations. The
SDP must describe the approach for performing the software
installation planning. This activity involves the preparation
for, and the installation and checkout, of the executable software at a user site. Planning and preparation should start
relatively early in the life cycle to ensure a smooth transition 


\end{multicols}
\begin{table}[h]
	\centering
	\setlength{\leftmargini}{0.5cm}
	\begin{tabular}{| m{8cm} | m{8cm} |}
		\hline
		\textbf{\textbf{Entry Criteria}} & \textbf{Exit Criteria}\\
		\hline
		\begin{itemize}
			\item The appropriate STP data items and
			other reference materials are obtained.
			\item Software requirements are established
			in the Software Requirements
			Specification and Interface
			Requirements Specification (IRS) and
			are traceable to a parent requirement.
			\item The top-level software architecture is
			established.
			\item The Requirements Test Verification
			Matrix (RTVM) specifies the test
			verification method and level for each
			requirement in the SRS and IRS. (See
			11.1.2).
		\end{itemize} &
		The following tasks have been defined and documented in the STP:
		\begin{itemize}
			\item Test environment (sites, hardware, software, test tools, test facilities,
			test data, etc.) needed to conduct the life cycle tests.
			\item Test scenarios to be performed including the schedule for executing
			the test activities.
			\item Traceability between SI requirements and the related tests and test
			phases where the requirements are verified.
			\item Personnel, organizations, responsibilities and management activities
			needed to develop and implement the planned testing.
			\item The objectives for each test including test level, type, conditions, data
			to be recorded, qualification method(s), data analysis, assumptions and
			constraints, safety, security and privacy considerations.
			\item Approach to related issues such as data rights, training, Regression
			Testing, delayed functionality and deliverable documentation.
			\item Criteria for evaluating the test results.
		\end{itemize} \\
		\hline
		\multicolumn{2}{|l|}{\textbf{Verification Criteria}}\\
		\hline
		\multicolumn{2}{|l|}{aaaaa}\\
		\hline
		\multicolumn{2}{|l|}{\textbf{Measurements}}\\
		\hline
		\multicolumn{2}{|l|}{\textbf{Statistics from the STP Peer Review. (See Chapter 15)}}\\
		\hline
	\end{tabular}
\end{table}
	
	\end{document}